\documentclass[12pt, letterpaper]{article}
\usepackage[utf8]{inputenc}
% math fonts
\usepackage{amssymb}
\usepackage{amsthm}
\usepackage{amsmath}

\title{Computability via Recursive Functions}
\author{Justin Pumford  }
\date{March 2020}

\newtheorem{theorem}{Theorem}
\newtheorem{lemma}{Lemma}
\newtheorem*{definition}{Definition}

% TODO: switch n and m around
\begin{document}
  \maketitle

  \section{Effective Calcubility and Computability}

  \section{Primitive Recursive Functions}
    \subsection{Functions}
      For this paper, $\mathbb{N}$ refers to the set $\{0, 1, 2, 3, ...\}$

      % zero function MUST accept one argument for further proofs to work

  \section{The Ackermann Function}
    % nested loop
    \begin{definition}[The Ackermann Function]
      Let $n, m \in \mathbb{N}$. Then define $A(n, m)$ as follows:
      \begin{equation*}
        \begin{aligned}
          A(m, n) &=
          \begin{cases}
            n + 1                   & m = 0 \\
            A(m - 1, 1)             & m > 0 \wedge n = 0 \\
            A(m - 1, A(m, n - 1))   & m > 0 \wedge n > 0
          \end{cases}
        \end{aligned}
      \end{equation*}
    \end{definition}

    \begin{lemma}
      \label{inn}
      For any $m, n \in \mathbb{N}, A(m, n) \in \mathbb{N}$
    \end{lemma}
    \begin{proof}
      Proof Here
    \end{proof}

    \begin{theorem}
      A(m, n) is a total function
    \end{theorem}

    \begin{proof}
      We will proceed inductively to show that $A(m, n)$ is defined for all $m, n \in \mathbb{N}$.
      \\ 
      \\
      Clearly $A(0, n)$ is defined for all $n \in \mathbb{N}$.
      Assume $A(k, n)$ is defined for some $k \in \mathbb{N}$ and every $n \in \mathbb{N}$.
      Since $k + 1 > 0$, $A(k + 1, 0) = A(k, 1)$, which is defined.
      \\
      \\
      Now we assume $A(k + 1, j)$ is defined for some $j \in \mathbb{N}$.
      By Lemma \ref{inn}, $A(k + 1, j) = a$ for some $a \in \mathbb{N}$.
      Then since $j + 1 > 0$, $A(k + 1, j + 1) = A(k, A(k + 1, j)) = A(k, a)$.
      Since $A(k, n)$ is defined for every $n \in \mathbb{N}$ by our inductive hypothesis,
      $A(k, a) = A(k + 1, j + 1)$ is defined.
    \end{proof}

    \begin{theorem}
      \label{lt}
      For any $m, n, s \in \mathbb{N}$ where $s > n$, $A(m, n) < A(m, s)$
    \end{theorem}
    \begin{proof}
      Use the proof of A(m, n) < A(m, n + 1)
    \end{proof}

    \begin{theorem}
      \label{plus2}
      For any $m, n, s \in \mathbb{N}$, $A(m, A(s, n)) < A(m + s + 2, n)$
    \end{theorem}
    \begin{proof}
      Proof here
    \end{proof}
    
    \begin{definition}
      Let $P$ be the set of all primitive recursive functions so that if $f(x_1, x_2, ..., x_n) \in P$ and
      $m = max\{x_1, x_2, ..., x_n\}$, then there exists $t \in \mathbb{N}$ so that 
      $f(x_1, x_2, ..., x_n) < A(t, m)$
    \end{definition}

    \begin{theorem}
      $c(x)$, $s(x)$, $p_i(x_1, x_2, ..., x_n) \in P$
    \end{theorem}
    \begin{proof}
      \begin{equation*}
        \begin{aligned}
          c(x) = 0 < x + 1 &= A(0, x) \\
          s(x) = x + 1 < x + 2 &= A(1, x) \\
          p_i(x_1, x_2, ..., x_n) = x_i \leq m < m + 1 &= A(0, m) \\
        \end{aligned}
      \end{equation*}
      To verify $x + 2 = A(1, x)$, we proceed by induction. \\
      $A(1, 0) = A(1 - 1, 1) = A(0, 1) = 2 = 0 + 2$. Now assume $A(1, k) = k + 2$ for some $k \in \mathbb{N}$.
      Then $A(1, k + 1) = A(1 - 1, A(1, k + 1 - 1)) = A(0, A(1, k)) = A(0, k + 2) = k + 3 = (k + 1) + 2$.
    \end{proof}

    \begin{theorem}
      $P$ is closed under composition
    \end{theorem}
    \begin{proof}
      Let $f, g_1, g_2, ..., g_k \in P$, where $f$ is $k$-ary and each $g_i$ is $j$-ary.
      Let $x_1, x_2, ..., x_j \in \mathbb{N}$.
      Let $m = max\{x_1, x_2, ..., x_j\}$.
      Let $h$ be the $j$-ary primitive recursive function that results from function composition of $f$ with $g_1, g_2, ..., g_k$.
      Let $g_{max}$ be the $g_i$ giving the maximum value in $max\{g_1(x_1, ..., x_j), ..., g_k(x_1, ..., x_j)\}$.
      Let $m_g = g_{max}(x_1, ..., x_j)$
      Since $g_{max} \in P$, there exists some $t_g \in \mathbb{N}$ so that $m_g < A(t_g, m)$.
      Similarly since $f \in P$, there exists some $t_f \in \mathbb{N}$ so that $h(x_1, ..., x_j) = f(g_1(x_1, ..., x_j), ..., g_k(x_1, ..., x_j)) < A(t_f, m_g)$.
      But since $m_g < A(t_g, m)$, by Theorem \ref{lt} $A(t_f, m_g) < A(t_f, A(t_g, m))$.
      By Theorem \ref{plus2}, $A(t_f, A(t_g, m)) < A(t_f + t_g + 2, m)$.
      Let $t = t_f + t_g + 2 \in \mathbb{N}$.
      Then $h(x_1, .., x_j) < A(t, m)$.
      So $h \in P$.
    \end{proof}

    \begin{theorem}
      $P$ is closed under primitive recursion
    \end{theorem}
    \begin{proof}
      
    \end{proof}

    \begin{theorem}
      $P$ is precisely the primitive recursive functions
    \end{theorem}
    \begin{proof}
    \end{proof}

    \begin{theorem}
      $A(m, n)$ is not a primitive recursive function
    \end{theorem}
    \begin{proof}
      Proof Here
    \end{proof}


  \section{General Recursive Functions}
    \subsection{Partial Functions}
    \subsection{Definition of General Recursive Functions}

\end{document}